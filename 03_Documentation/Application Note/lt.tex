\subsubsection{Erster Ansatz}
  Als ersten Ansatz w�hlten wir die naheliegende Idee die Exportfunktion von Matlab/Simulink Modellen zu legitimem C-Code zu nutzen. Hierbei lie� sich die C2000-Architektur ausw�hlen und augenscheinlich korrekte Abbildungen der Simulationsbl�cke wurden in einzelnen C-Sourcecodedateien erstellt. Die Bl�cke wurden modular mit einem Eingangs- und Ausgangsstruct des jeweiligen, meist etwas kryptischen Typs, versehen, deren Member den Datenfl�ssen aus der Simulation entsprechen. Jedes dieser Modelle beinhaltete zudem eine Funktion {\lstinline[breaklines=true]$<Modellname>_step$}, die einen Simulationsschritt darstellt und so bei uns nach Konfiguration des entsprechenden Schrittintervalls in einer Timerroutine benutzt worden w�re. Final scheiterte der Ansatz jedoch daran, dass das elektrische Modell des Motors nicht nachgestellt und exportiert werden konnte und so der komplexeste Teil der Simulation noch immer �brig geblieben w�re. Diese kurze Ausf�hrung soll daher gern als Abschreckung gesehen werden und zur Vorsicht in Bezug auf modellgenerierten Code f�r die C2000-Reihe aufrufen.

\subsubsection{Tats�chliche Realisierung}
  Die tats�chlich genutzte Implementierung benutzt 6 PWMs aus dem TI-ePWM-Modul \todo{link zu http://www.ti.com/lit/ug/spruge9e/spruge9e.pdf .. wie am besten?}, durch deren Dutycycle-Ver�nderung �ber Zeit ein Pulsmuster erzeugt wird, mit dem die 6 MosFETs so beschaltet werden, dass der Motor sich dreht. Dies geschieht durch ein sukzessives Inkrementieren einer Z�hlvariablen in der ISR-Routine jedes der drei EPWM-Handlers (alle 0.4ms). Diese Z�hlvariable dient dann als Index f�r ein vorgeneriertes Array mit Sinuswerten, dezimal normiert auf +/- 1600 (2000 ist der eingestellte Registerwert, der die Periode darstellt, eine 1600 steht bspw. f�r einen Duty Cycle von 20\% (1-(1600/2000))). Der Wert war bewusst nicht auf die volle Aussteuerung ausgelegt, um als Test einen gem��igten Betrieb zu nutzen und �bermodulation beziehungsweise Blocktaktung zu vermeiden. Beim �bertragen der Werte aus dem Sinus-Array wird also der Reload-Wert des jeweiligen ePWM-Registers gesetzt, der damit die Zeit f�r beide logischen Zust�nde festlegt. Da es sich um 6 PWMs handelt, verhalten sich jeweils zwei zueinander entgegengesetzt und diese 3 Paare wiederum um ein Drittel verschoben voneinander.  
 
 \subsubsection{Hinweise zum CCS-Projekt}
  Das Makefile des Demoprogramms, das wir als Framework nutzten, sucht teilweise fehlerhaft \todo{link zu http://e2e.ti.com/support/microcontrollers/c2000/f/171/p/326227/1136098 .. wie am besten?} an absoluten Pfaden, zum Beheben haben wir einen symbolischen Link auf das "angebliche" Verzeichnis gesetzt: {\lstinline[breaklines=true]$mklink /D "C:\TI\controlSUITE2_DMC Rev" "C:\Beuth\ti_controlSUITE\"$}. Das Projekt nutzt zahlreiche Includes, teiweise aus DSP2803x\_headers und DSP2803x\_common, aber h�ufig auch aus den TI-Installationsverzeichnissen beziehungsweise wieder absolute Pfade, daher empfiehlt es sich wahrscheinlich meistens TI-Produkte nur mit Standardeinstellungen (ggf. in VM) zu installieren. 
 