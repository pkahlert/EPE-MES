\subsubsection{erster Ansatz}
  - Matlab importieren
  - gescheitert an elektr. Modell und dependency-Problemen
  - wie in Simulink feldorientiert regeln mit Drehzahl als input und Ausgabe der sechs PWM-Duty-Cycles die die Mosfets beschalten
  

\subsubsection{tats�chliche Realisierung}
 - Inkrementieren einer Z�hlvariable
  - Nutzen der Z�hlvariable als Index f�r ein vorgeneriertes Array mit Sinuswerten
  - diese Sinuswerte werden jeweils um ein Drittel verschoben in Form von Duty-Cycles auf die sechs PWMs gegeben, bei denen jeweils zwei zu Schalter-Paaren zusammengefasst und sich zueinander entgegengesetzt verhalten
  - die Duty-Cycle-Werte sind Dezimal bezogen auf eine festgelegte Periode von 2000 ausgerechnet
  - Beim �bertragen der Werte wird so der Reload-Wert des jeweiligen ePWM-Registers gesetzt, der damit die Zeit f�r beide logischen Zust�nde festlegt
  - unmodifiziert w�re das der Betrieb auf fester Drehzahl mit Stellreserve (nicht �bermoduliert)
 
 \subsubsection{Hinweise zum CCS-Projekt}
 # TI's Fehler ausbaden

Das Makefile des Demoprogramms nutzt teilweise fehlerhaft (s. http://e2e.ti.com/support/microcontrollers/c2000/f/171/p/326227/1136098 ) absolute Pfade, zum Beheben haben wir einen symbolischen Link auf das "angebliche" Verzeichnis gesetzt:

```cmd
mklink /D "C:\TI\controlSUITE2_DMC Rev" "C:\Beuth\ti_controlSUITE\"
```